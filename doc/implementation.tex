As mentioned earlier, OTPO is an Open MPI specific tool and is implemented in C. OTPO reads a file supplied by the
user that contains all the parameter names that the user wants to test and their values in a specific format. OTPO then executes an MPI job specified as a startup argument for (all) different combinations of attribute values. A metric provided by the MPI job, e.g. network latency or execution time, is returned to OTPO and used as the measure in order to determine which sets of MCA parameters lead to the best performance. As of now, OTPO only supports measuring the network latency using the Netpipe\cite{netpipe} benchmark. However, since the overall architecture of OTPO is very modular, it is straight forward to add and use another application. At the end of the run, OTPO outputs a list of the best parameter combinations for that test.

OTPO is meant to run on the head node of a cluster, and it forks MPI jobs with the desired application (e.g. Netpipe) after exporting the current combination of MCA parameters on the nodes. Internally, OTPO is built on top of ADCL \cite{ADCL} (Abstract Data and Communication Library) for maintaining parameter values, choosing the next parameter combination to be tested etc. In the following, we will detail the most relevant aspects of ADCL for OTPO as well as the interactions between OTPO, ADCL and MPI.

\subsection{ADCL}
The Abstract Data and Communication Library is an adaptive communication library. The goal of ADCL is to optimize application level communication patterns such as n-dimensional neighborhood communication or shift operations for a given execution environment. The library provides for each (predefined) communication pattern a large number of implementations and incorporates a runtime selection logic in order to choose the implementation leading to the highest performance of the application on the current platform. However, an application also has the possibility to register its own functions with ADCL and use the ADCL internal runtime selection logic to determine, which of the alternative implementations provided lead to the best overall performance. This mechanism has been used by OTPO for registering and maintaining the status of different MCA parameters. Therefore, we detail the semantics of ADCL in the following paragraphs:

\begin{itemize}
\item ADCL\_Attribute: an abstraction for a particular
  characteristic of a function. Each attribute is represented by the set of 
  possible values for this characteristic. In OTPO, a separate ADCL\_Attribute
  objects is created for each MCA parameter and its possible  values.
\item ADCL\_Attrset: is a set of ADCL attributes. Thus, an 
  attribute set is the structure holding all MCA parameters provided by the user in the input
  and file and their possible values.
\item ADCL\_Function: equivalent to an actual implementation
  of a particular communication pattern. In OTPO an ADCL\_Function is an
  object that holds one possible combination of MCA parameter values.
\item ADCL\_Fnctset: is a set of ADCL functions providing the same
	functionality. An ADCL\_Fnctset can have an ADCL\_Attrset attached to it. 
\item ADCL\_Request: a user level handle combining all required information to initiate a 
	communication by starting a request. We will detail the usage of this object in OTPO  
  in the following paragraphs. 
\end{itemize}

ADCL provides as of now two different runtime selection algorithms, namely a {\it brute force search} which tests all available implementations, and a {\it heuristic search} which optimizes individual attributes used to characterize an implementation. The main disadvantage of the brute force search strategy is, that even in cases where a particular parameter value has shown to cause bad results, all possible combinations of attributes will be (unnecessarily) tested, which might take a long time. In order to reduce the execution time of the selection logic, the heuristic based approach tests attribute combinations by modifying only one attribute at a time. After all possible values for a single attribute have been tested and evaluated, the value for that attribute leading to the best performance is determined by the library. In the subsequent optimization of the next attribute, only the 'optimal' values for the parameters optimized so far will be used. This approach reduces the number of attribute combinations which have to be evaluated dramatically. It has however the restriction, in that it assumes, that different attributes used to characterize an implementation are not correlated. In order to overcome this limitation further runtime selection algorithms are currently being implemented.

Although OTPO has been using mostly the brute force search strategy so far, one of the advantages of using ADCL within OTPO relies on the possibility to seamlessly switch the selection logic used underneath, which determined the next parameter combination to be tested.


%ADCL is an add-on library to MPI, but as we stated earlier, OTPO will not be an MPI
%application. OTPO will always be a single process application and will not use
%any part of ADCL where MPI is needed. In order to achieve that, we created a
%dummy-mpi library within ADCL that can be enabled on configure time. This dummy
%library is just meant to be there to allow ADCL to compile without the real
%MPI library. The dummy library basically defines MPI parameters to dummy
%values and MPI functions to do nothing. Another reason for the dummy MPI
%library is the fact that MPI and fork cause badness in the application.

\subsection{OTPO parameter file and runtime options}

The OTPO parameter file describes the MCA parameters and potential values to be tested. In order to provide a maximum flexibility to the end-user, the parameters can be described by various options, e.g. depending on whether a parameter can have continues values, certain discrete values, or whether the value consists of different strings. Each line in the parameter file specifies a single parameter by giving the name of the parameter and some options, the options being one of the following:
\begin{itemize}
\item d default value
\item p {possible\_values}:option for the user to explicitly list the possible
  values for the parameter
\item r start\_value end\_value: specify the start and end value for the
  parameter.
\item t traversal method arguments: The method to traverse the range of
  variables for the parameter. The increment method is only available now,
  which takes the operation and the operator as arguments.
\item i rpn: RPN (reverse polish notation) condition that the parameter
  combinations must satisfy.
\item v: specifies if the parameter is virtual, which means that it won't be
  set as an environment variable, but will be part of another parameter.
\item a {format string}: specifies that the parameter is an aggregate of
  other parameters in a certain format. Each sub parameter is surrounded by \$
  signs in the format string.
\end{itemize}

After the user creates the input file, OTPO can be run using one of the following options:
\begin{itemize}
\item {\tt p} InputFileName (file that contains the parameters) *
\item {\tt d} (debug output)
\item {\tt v} (verbose output)
\item {\tt s} (status output)
\item {\tt n} (silent/no output)
\item {\tt t} test (name of test, Current supported: Netpipe)
\item {\tt w} test\_path (path to the test on your system) *
\item {\tt l} message\_length (default is 1 byte)
\item {\tt h} hostfile
\item {\tt m} mca\_params (mca parameters that you want your mpirun to run with)
\item {\tt f} format (format of output, TXT)
\item {\tt o} output\_dir (directory where the results will be placed, default: results)
\item {\tt b} interrupt\_file (file to write data to when interrupted, default:
  interrupt.txt)
\item {\tt r} interrupt\_file (the file where contains the data to resume execution) 
\end{itemize}

\subsection{Interaction between OTPO, ADCL and the MPI library}

In the following, we detail the sequence of events for an OTPO execution including how OTPO utilizes the ADCL library and executes the MPI application. OTPO parses the input file, using simple string tokenizing techniques, and creates a
global structure that holds all the parameters and their options. This
structure will be used to generate the ADCL objects. 

An array of ADCL attributes with size equals to the number of the parameters
is allocated. Each attribute is created by generating all the possible values
corresponding to that attribute, using the global structure. Then an attribute
set is created from the array of attributes. The next step is to create an
ADCL function object for each possible combination which satisfies all the RPNs specified in
the parameter file. The array of function objects is then used to create the ADCL function set. 
Finally, an ADCL request object is created to start the measurements.

Once the ADCL request object is created, the request is 'started' for the given number of parameter value combinations.
Starting the request is equivalent with calling a function registered with the according ADCL\_Fnctset. The function registered by OTPO performs the following actions:
\begin{itemize}
\item Fork a child process
\item The child process sets the arguments that needs to be provided to the {\tt mpirun}
  command, such as the number of processes, MCA parameters and values, and the application/benchmark to run.
%\item The child then gets the current function object of the request with a
%  value for each parameter.
%\item The child exports the parameter values to the environment.
\item Finally the child executes (using execvp) {\tt mpirun} with the argument
  set.
\item The parent process checks if the child was successful (the execvp succeeded)
\item If the child succeeded, the parent will update the current request with
  the value (ex, latency) that the child measured for the current parameter values.
\item If the child did not complete after some time, the parent process kills it, and
  updates the request with an invalid value.
\end{itemize}

When all measurements for all combinations of parameter values have been updated to the ADCL request object, 
OTPO gathers the results and places them in a sub-directory. Each run of OTPO produces a
file with a time stamp that contains the best attribute combinations. The
result file contains the set of best measured values, the number of combinations that
produced these value, and the parameter value combinations themselves. The result file might be very large,
having thousands of different parameter combinations. 

These results files produced by OTPO are meant to be intermediate results to
an analysis tool in OTPO that takes in any number of result files, performs some
analysis on them presents a final analysis in a more user-friendly form. The analysis
option in OTPO is however still under development and research.

%%% Local Variables: 
%%% mode: latex
%%% TeX-master: "paper"
%%% End: 
