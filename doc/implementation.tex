As mentioned earlier, OTPO is an Open MPI specific tool. This means that Open
MPI must be installed before configuring and compiling OTPO. OTPO however is
not an MPI application. OTPO is built on top of ADCL \cite{ADCL} (Abstract
Data and Communication Library). In this section, we will shed some light on
what ADCL is and its functionality, then we will talk about how OTPO is
implemented and what is still missing.

\subsection{ADCL}
The Abstract Data and Communication Library is an application level
communication library aiming at providing the highest possible performance for
application level communication patterns within a given execution
environment. A communication pattern is a defined as a repeatedly occuring
communication operation incorporating a group of processes. The library
provides for each communication pattern a large number of implementations and
incorporates a runtime selection logic in order to choose the implementation
leading to the highest performance of the application in the current
platform. ADCL right now provides two runtime selection algorithms:
\begin{enumerate}
\item Brute Force Strategy: tests all available implemetations
\item Heuristic Strategy: relying on attributes characterizing an
  implemetation.
\end{enumerate}
In OTPO, the brute force strategy will be used, since we are interested in
testing all possible combinations of parameters.

Before we go into the details of how OTPO will use ADCL, we will explain the
ADCL objects that are needed for the OTPO implementation:
\begin{itemize}
\item ADCL\_Attribute\cite{ADCL-spec}: an abstraction for a particular
  charactersitic of a function. In OTPO, there will be an array of
  ADCL\_Attribute objects, each holding an MCA parameter name and its possible
  values.
\item ADCL\_Attrset\cite{ADCL-spec}: is a set of ADCL attributes.
\item ADCL\_Function\cite{ADCL-spec}: equivalent to an actual implementation
  of a particular communication pattern. In OTPO an ADCL\_Function is an
  object that holds one possible combination of values of the MCA
  parameters. Each ADCL\_Function will have an ADCL\_Attrset attached to it.
\item ADCL\_Fnctset\cite{ADCL-spec}: is a set of ADCL functions, which means
  it's the structure that holds all the possible combinations of the MCA
  parameter space provided by the user.
\item ADCl\_Request\cite{ADCL-spec}: a user level handle that allows the
  application to initiate a communication by starting a request. The use of
  this object in OTPO will be more clear later when the implementation of OTPO
  is explained.
\end{itemize}

ADCL is an MPI application, but as we stated earlier, OTPO will not be an MPI
application. OTPO will always be a single process application and will not use
any part of ADCL where MPI is needed. In order to achieve that, we created a
dummy-mpi library within ADCL that can be enabled on configure. This dummy
library is just meant to be there to allow ADCL to compile without the real
MPI library. The dummy library basically defines MPI parameters to dummy
values and MPI functions to do nothing. Another reason for the dummy MPI
library is the fact that MPI and fork cause badness in the application.

\subsection{OTPO}
OTPO is implemented in C. OTPO is inteneded to read a file supplied by the
user that contains all the parameter names that the user wants to test and
their attributes in a specific format. OTPO then executes an MPI job,
measuring latency from Netpipe\cite{netpipe}(the only measurement available
right now). However the tests used for the measurements are modular, so at any
point, another test measuring latency or something else like bandwidth can be
added. OTPO outputs a list of the best parameter combinations for that test.

OTPO is meant to run on the head node of a cluster, and it forks MPI jobs with
the desired application (Netpipe, latency) after exporting the current
combination of MCA parameters on the nodes.

The parameter file contains the parameter names each followed by some options:
\begin{itemize}
\item -d default value
\item -p {possible\_values}:option for the user to explicitly list the possible
  values for the parameter
\item -r start\_value end\_value: specify the start and end value for the
  parameter.
\item -t traversal method arguments: The method to traverse the range of
  variables for the parameter. The increment method is only available now,
  which takes the operation and the operator as arguments.
\item -i rpn: RPN (reverse polish notation) condition that the parameter
  combinations must satisfy.
\item -v: specifies if the parameter is virtual, which means that it won't be
  set as an environment varaible, but will be part of another parameter.
\item -a {format string}: specifies that the parameter is an aggregate of
  other parameters in a certain format. each subparameter is surrounded by \$
  signs in the format string.
\end{itemize}

After the user creates this file, OTPO will be ready to run. The usage options
for running OTPO are:
\begin{itemize}
\item -p InputFileName (file that contains the parameters) *
\item -d (debug output)
\item -v (verbose output)
\item -s (status output)
\item -n (silent/no output)
\item -t test (name of test, Current supported: Netpipe)
\item -w test\_path (path to the test on your system) *
\item -l message\_length (default is 1 byte)
\item -h hostfile
\item -m mca\_params (mca parameters that you want your mpirun to run with)
\item -f format (format of output, TXT)
\item -o output\_dir (directory where the results will be placed, default: results)
\item -b interrupt\_file (file to write data to when interrupted, default:
  interrupt.txt)
\item -r interrupt\_file (the file where contains the data to resume execution) 
\end{itemize}

OTPO parses the file, using simple string tokenizing techniques, and creates a
global structure that holds all the parameters and their options. This
structure will be used to generate the ADCL objects. 

An array of ADCL attributes with size equals to the number of the parameters
is allocated. Each attribute is created by generating all the possible values
corresponding to that attribute, using the global structure. Then an attribute
set is created from the array of attributes. The next step is to create an
ADCL function object for each possible combination. So to do that, all the
possible combinations are genereted, but before the function object is
created, each combination is checked if it satisfies all the RPNs specified in
the parameter file. If that's the case, then we create the ADCL function
object for that combination, specifying at that point the pointer to the test
function to call. An array of all the function objects that were created is
used to create the function set. Finally, an ADCL request object is created to
start the measurements.

Now that we have an ADCL request object created, we start a request for the
number of combination. Each request calls the test function pointer that was
registered for the function object. The test function does the following:
\begin{itemize}
\item Fork off a child
\item The child sets the arguments that needs to be provided to the mpirun
  commnad, like the number of processes, mca parameters, and the test to run
  (ex. Netpipe).
\item The child then gets the current function object of the request with a
  value for each parameter.
\item The child exports the parameter values to the environment.
\item Finally the child executes (using execvp) the mpirun with the argument
  set.
\item The parent checks if the child was successfull (the execvp succeeded)
\item If the child succeeded, the parent will update the current request with
  the value (ex, latency) that the child measured when the current parameters
  were set.
\item If the child didn't complete after some time, the parent kills it, and
  updates the request with an invalid value.
\end{itemize}

When all the requests are executed, and the measurements for all the
combinations have been updated to the ADCL request object, OTPO gathers the
results and place them in a sub-directory. Every single run of OTPO produces a
file with a time stamp that contains the best attribute combinations. It gives
the best combination around the best value that it found. 

These results that OTPO produce consist of the combinations that
were found to be the best for the specified test on the specified machine. The
result file contains all the best measured values, the number of combinations that
produced these value, and the combinations themselves. This is surely not a
readable end result to the user, since the result file might be very large,
where we have thousands of different combinations. 

These results files produced by OTPO are meant to be intermediate results to
an analysis tool in OTPO that takes in any number of result files, does some
sort of analysis, and gives the final analysis to the user. The analysis
option in OTPO is still under development and research.

%%% Local Variables: 
%%% mode: latex
%%% TeX-master: "paper"
%%% End: 
