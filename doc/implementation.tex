%OTPO reads a user-supplied file that contains all the MCA parameter
%names under test, as well as the ranges of values for each parameter.
%OTPO then executes an MPI benchmark job for all combinations of the
%parameter values. A metric provided by the benchmark, such as latency
%or wall-clock execution time, is returned to OTPO and used as the
%measure to determine which sets of MCA parameters lead to the best
%performance. After all the parameters have been tested, OTPO outputs a
%list of the parameter combinations that resulted in the best
%performance for that test.

Upon start of an optimization run, OTPO parses an input file %using simple string tokenizing techniques 
and creates a global structure that holds all the
parameters and their options. OTPO than register a function for each possible combination of MCA parameters which satisfy the RPNs specified in the parameter file with ADCL. The function registered by OTPO first forks a child process. The child process sets the parameters in the environment that
need to be provided to the {\tt mpirun} command, such as the number of processes, MCA parameters and values, and the   application/benchmark to run. Finally, the child launches {\tt mpirun} with the argument set. The parent process waits for the child to complete, and checks if the test was successful. %(e.g., {\tt mpirun} completed with an exit status of 0)
If the child succeeded, the parent will update the current request with the value (e.g., latency) that the child measured for the current parameter values. If the child does not complete within a specified timeout, the parent process kills it, and updates the request with an invalid value.

When measurements for all combinations of parameter values have been
updated by ADCL, OTPO gathers the results and saves
them to a file. Each run of OTPO produces a file with a time stamp
that contains the best attribute combinations. The result file
contains the set of best measured values, the number of combinations
that produced these values, and the parameter value combinations
themselves. The result file might be large, having thousands of
different parameter combinations.

These results files produced by the first version of OTPO are intented
to be intermediate results.  Work is ongoing to research and design a
tool that can present the results in an intuitive and visual manner.


\subsection{OTPO Parameter File}

The OTPO parameter file describes the MCA parameters and potential
values to be tested. In order to provide a maximum flexibility to the
end-user, the parameters can be described by various options,
e.g. depending on whether a parameter can have continues values,
certain discrete values, or whether the value consists of different
strings. Each line in the parameter file specifies a single parameter
by giving the name of the parameter and some options, the options
being one of the following:
\begin{itemize}
\item {\tt d default\_\-value}: specifies a default value for this
  parameter.
\item {\tt p <list of possible values>}: explicitly specify the list
  of possible values for the parameter.
\item {\tt r start\_\-value end\_\-value}: specify the start and end
  value for the parameter.
\item {\tt t traversal\_\-method <arguments>}: specifies the method to
  traverse the range of variables for the parameter. The first version
  of OTPO only includes one method: ``increment,'' which takes the
  operator and the operand as arguments.
\item {\tt i rpn\_\-expression}: RPN (reverse polish notation)
  condition that the parameter combinations must satisfy.
\item {\tt v}: specifies if the parameter is virtual, which means that
  it will not be set as an environment variable, but will be part of
  another parameter.
\item {\tt a format\_\-string}: specifies that the parameter is an
  aggregate of other parameters in a certain format. Each sub
  parameter is surrounded by dollar signs (\$) in the format string.
\end{itemize}

%After the user creates the input file, OTPO can be run using one of the following options:
%\begin{itemize}
%\item {\tt p InputFileName}: file specifying parameters and values to test.
%\item {\tt d}: enable debuging output.
%\item {\tt v}: enable verbose output.
%\item {\tt s}: enable periodic progress status messages.
%\item {\tt n}: silent; disable all output.
%\item {\tt t test}: name of test to use (currently, only ``Netpipe'').
%\item {\tt w test\_\-path}: path to the test executable on your
%  system.
%\item {\tt l message\_\-length} length of messages to test (default is
%  1 byte).
%\item {\tt h hostfile}: hostfile to use with {\tt mpirun}.
%\item {\tt m mca\_\-params} Additional MCA parameters to pass to {\tt
%    mpirun}.
%\item {\tt f format}: format of the output (currently, only ``text'').
%\item {\tt o output\_\-dir}: directory where the results will be
%  placed (default: results).
%\item {\tt b interrupt\_\-file}: file to write data to when
%  interrupted (default: interrupt.txt).
%\item {\tt r interrupt\_\-file}: read from a previously-written
%  ``interrupted'' data file; execution will resume where it left off.
%\end{itemize}

%%% Local Variables: 
%%% mode: latex
%%% TeX-master: "paper"
%%% End: 
