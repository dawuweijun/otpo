Open MPI~\cite{gabriel:ompi} is an open source implementation of the
MPI-1~\cite{mpi1} and MPI-2~\cite{mpi2} specifications. The code is
developed and maintained by a consortium consisting of 14
institutions\footnote{As of January, 2008.} from academia and
industry. The Open MPI design is centered around the Modular Component
Architecture (MCA), which is the software layer providing management
services for Open MPI frameworks. A framework is dedicated to a single
task, such as providing collective operations (i.e., the COLL
framework) or providing data transfer primitives for a particular
network interconnect (i.e., the Byte Transfer Layer framework --
BTL). Each framework will typically have multiple implementations
available in the form of modules (``plugins'') that can be loaded
on-demand at run time.  For example, BTL modules include support for
TCP, InfiniBand~\cite{ib}, Myrinet, shared memory, and others.

Among the management services provided by the MCA is the ability to
accept run-time parameters from higher level abstractions (e.g., {\tt
  mpirun}) and pass them down to the according frameworks. MCA runtime
parameters give system administrators, end-users and developers the
possibility to tune the performance of their applications and systems
without having to recompile the MPI library. Examples for MCA runtime
parameters include the values of cross-over points between different
algorithms in a collective module, or modifying some network
parameters such as internal buffer sizes in a BTL module. Due to its
great flexibility, Open MPI developers made extensively use of MCA
runtime parameters.  The current development version of Open MPI has
multiple hundred MCA runtime parameters, depending on the set of
modules compiled for a given platform. While average end-users clearly
depend on developers setting reasonable default values for each
parameter, some end-users and system administrators might explore the
parameter space in order to find values leading to higher performance
for a given application or machine.

OTPO is a tool aiming at optimizing parameters of the runtime environment
exposed through the MPI library to the end-user application. OTPO is based on
a highly modular concept, giving end-user the possibility to provide or
implement their own benchmark for exploring the parameter space. Depending on
the purpose of the tuning procedure, most often only a subset of the runtime
parameters of Open MPI will be tuned at a given time. As an example, users
might choose to tune the networking parameters for a cluster, optimizing the
collective operations in a subsequent run etc. Therefore, one of the goals of
OTPO is to provide a flexible and user friendly possibility to input the set
of parameters to be tuned. 
OTPO supports testing two general types of MCA parameters:

\begin{enumerate}
\item Single-value parameters: these parameters represent an
  individual value, such as an integer.
\item Multiple-value parameters: these parameters are composed of
  one or more sub-parameters, each of which can vary.
\end{enumerate}

From a higher level perspective, the process of tuning runtime parameters is an empirical search in a given parameter space. Depending on the number of parameters, the number of possible values for each parameter, and dependencies among the parameters themselves, the tuning process can in fact be very time consuming and complex. In order to avoid replicating work by other researchers OTPO is based on the a library incorporating various search algorithms, namely the Abstract Data and Communication Library (ADCL)~\cite{ADCL}. ADCL is a runtime optimization library, which gives applications the possibility to register alternative implementations for a given functionality. Using different runtime decision algorithms, the library will evaluate the performance of each implementation provided, and choose the version leading to the lowest execution time. The application has furthermore the possibility to characterize each implementation using a set of attributes. These attributes are used by ADCL for runtime decisions algorithms which go beyond a simple brute force search strategy, e.g. optimizing each parameter individually. This mechanism has been used by OTPO for registering and maintaining the status of different MCA parameters. 