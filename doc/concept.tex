Open MPI~\cite{gabriel:ompi} is an open source implementation of the
MPI-1~\cite{mpi1} and MPI-2~\cite{mpi2} specifications. The code is
developed and maintained by a consortium consisting of 14
institutions\footnote{As of January, 2008.} from academia and
industry. The Open MPI design is centered around the Modular Component
Architecture (MCA), which is the software layer providing management
services for Open MPI frameworks. A framework is dedicated to a single
task, such as providing collective operations (i.e., the COLL
framework) or providing data transfer primitives for a particular
network interconnect (i.e., the Byte Transfer Layer framework --
BTL). Each framework will typically have multiple implementations
available in the form of modules (``plugins'') that can be loaded
on-demand at run time.  For example, BTL modules include support for
TCP, InfiniBand~\cite{ib}, Myrinet, shared memory, and others.

Among the management services provided by the MCA is the ability to
accept run-time parameters from higher level abstractions (e.g., {\tt
  mpirun}) and pass them down to the according frameworks. MCA runtime
parameters give system administrators, end-users and developers the
possibility to tune the performance of their applications and systems
without having to recompile the MPI library. Examples for MCA runtime
parameters include the values of cross-over points between different
algorithms in a collective module, or modifying some network
parameters such as internal buffer sizes in a BTL module. Due to its
great flexibility, Open MPI developers made extensively use of MCA
runtime parameters.  The current development version of Open MPI has
multiple hundred MCA runtime parameters, depending on the set of
modules compiled for a given platform. While average end-users clearly
depend on developers setting reasonable default values for each
parameter, some end-users and system administrators might explore the
parameter space in order to find values leading to higher performance
for a given application or machine.

OTPO is a tool aiming at optimizing parameters of the runtime environment exposed through the MPI library to the end-user application. OTPO is based on a highly modular concept, giving end-user the possibility to provide or implement their own benchmark for exploring the parameter space. Depending on the purpose of the tuning procedure, most often only a subset of the runtime parameters of Open MPI will be tuned at a given time. As an example, users might choose to tune the networking parameters for a cluster, optimizing the collective operations in a subsequent run etc. Therefore, one of the goals of OTPO is to provide a flexible and user friendly possibility to input the set of parameters to be tuned. 

From a higher level perspective, the process of tuning runtime parameters is an empirical search in a given parameter space. Depending on the number of parameters, the number of possible values for each parameter, and dependencies among the parameters themselves, the tuning process can in fact be very time consuming and complex. A large number of software projects are dealing with parameter optimization. In order to avoid replicating work which has been done by other researchers OTPO is based on the ADCL adaptive library~\cite{ADCL}. In the following, we outline the concept of ADCL. 

\subsection{ADCL}

The ADCL is an adaptive communication library that aims to optimize a
predefined set of application-level communication patterns such as
N-dimensional neighborhood communication or shift operations for a
given execution environment. The library provides for each
communication pattern a large number of implementations and
incorporates a runtime selection logic in order to choose the
implementation leading to the highest performance of the application
on the current platform. However, an application also has the
possibility to register its own functions with ADCL and use the ADCL
internal runtime selection logic to determine, which of the
alternative implementations provided lead to the best overall
performance. This mechanism has been used by OTPO for registering and
maintaining the status of different MCA parameters. The main 
ADCL objects within the ADCL API are:

\begin{itemize}
\item {\tt ADCL\_\-Attribute:} an abstraction for a particular
  characteristic of a function. Each attribute is represented by the
  set of possible values for this characteristic. In OTPO, a separate
  {\tt ADCL\_\-Attribute} objects is created for each MCA parameter
  and its possible values.
\item {\tt ADCL\_\-Attrset:} a set of ADCL attributes. Thus, an
  attribute set is the structure holding all MCA parameters provided
  by the user in the input and file and their possible values.
\item {\tt ADCL\_\-Function:} equivalent to an actual implementation
  of a particular communication pattern. In OTPO, an {\tt
    ADCL\_\-Function} is an object that holds one possible combination
  of MCA parameter values.
\item {\tt ADCL\_\-Fnctset:} is a set of ADCL functions providing the
  same functionality. An {\tt ADCL\_\-Fnctset} can have an {\tt
    ADCL\_\-Attrset} attached to it.
\item {\tt ADCL\_\-Request:} a user level handle combining all
  required information to initiate a communication by starting a
  request. We will detail the usage of this object in OTPO below.
\end{itemize}

ADCL currently provides two different runtime selection algorithms: a
{\it brute force search} which tests all available implementations,
and a {\it heuristic search} which optimizes individual attributes
used to characterize an implementation. The main disadvantage of the
brute force search strategy is that even parameter value sets that are
known to not provide optimized behavior will be (unnecessarily)
tested, thereby potentially greatly increasing the overall required
time for testing. In order to reduce the execution time of the selection
logic, the heuristic based approach tests attribute combinations by
modifying only one attribute at a time. After all possible values for
a single attribute have been tested and evaluated, the value for that
attribute leading to the best performance is determined by the
library. In the subsequent optimization of the next attribute, only
the ``optimal'' values for the parameters optimized so far will be
used. This approach can dramatically reduce the number of attribute
combinations that are evaluated. However, it has the restricting
assumption that different attributes used to characterize an
implementation are not correlated. In order to overcome this
limitation, further runtime selection algorithms are currently being
implemented.