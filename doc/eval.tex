This section presents an example using OTPO to optimize some of the
InfiniBand parameters of Open MPI on a given platform.  We therefore
first describe Open MPI's InfiniBand support and some of its run-time
tunable parameters, then present the results of the optimization using
OTPO.

\subsection{InfiniBand Parameters in Open MPI}

\label{sec:motivation}

Open MPI supports InfiniBand networks through a Byte Transfer Layer
(BTL) plugin module named {\tt openib}.  BTL plugins are the lowest
layer in the Open MPI point-to-point communication stack and are
responsible for actually moving bytes from one MPI process to
another.  The {\tt openib} BTL has both single- and multiple-value
parameters that can be adjusted at run-time.

%Open MPI internally ranks network interconnects based on the best
%latency achievable by that network.  Upon application startup, the MPI
%library determines the list of available networks on each process and
%to each process.  A single process may therefore use multiple network
%interconnects for communication to different processes.  The default
%behavior of Open MPI always chooses the network with the lowest
%latency for short messages.  The ``{\tt btl}'' MCA run-time parameter
%can be used to override this behavior and manually enforce the usage
%of a particular network (or set of networks) for MPI communications,
%For example, to force the use of InfiniBand networks, a user can
%specify {\tt --mca btl openib} on the command line, indicating that
%{\em only} the {\tt openib} BTL module should be used.  Note that this
%will disable all other BTL modules, even potentially lower latency
%networks (such as shared memory, which would have otherwise been used
%for communication on the same node).

There are more than 50 MCA parameters that are related to the {\tt
  openib} BTL module, all of which can be modified at runtime.  Open
MPI attempts to provide reasonable default values for these
parameters, but every application and every platform is different:
maximum performance can only be achieved through tuning for a specific
platform and application behavior.

%
%The following are some of the {\tt openib} BTL's single-value
%parameters that are used to tune Open MPI's short message RDMA
%behavior:\footnote{Note that the exact definition and usage of these
%  parameters are outside the scope of this paper; they are only
%  briefly described here.}

%\begin{itemize}
%\item {\tt btl\_\-openib\_\-ib\_\-max\_\-rdma\_\-dst\_\-opts}: maximum
%  number of outstanding RDMA operations to a specific destination.
%\item {\tt btl\_\-openib\_\-use\_\-eager\_\-rdma}: a logical value
%  specifying whether to use the RDMA protocol for eager messages.
%\item {\tt btl\_\-openib\_\-eager\_\-rdma\_\-threshold}: only use RDMA
%  for short messages to a given peer after this number of messages has
%  been received from that peer.
%\item {\tt btl\_\-openib\_\-max\_\-eager\_\-rdma}: maximum number of
%  peers allowed to use RDMA for short messages.
%\item {\tt btl\_\-openib\_\-eager\_\-rdma\_\-num}: number of RDMA
%  buffers to allocate for short messages.
%\end{itemize}

MPI processes communicate on InfiniBand networks by setting up a pair
of queues to pass messages: one queue for sending and one queue for
receiving.  InfiniBand queues have a large number of attributes and
options that can be used to tailor the behavior of how messages are
passed.  Starting with version v1.3, Open MPI exposes the receive
queue parameters for short messages through the multiple-value
parameter {\tt btl\_\-openib\_\-receive\_\-queues} (long messages use
a different protocol and are governed by a different set of MCA
parameters).  Specifically, this MCA parameter is used to specify one
or more receive queues that will be setup in each MPI process for
InfiniBand communication.  There are two types of receive queues, each
of which have multiple sub-parameters. It is however outside of the scope of this paper to give detailed and precise descriptions of the MCA parameters used. The parameters are:

\begin{enumerate}
\item ``Per-peer'' receive queues are dedicated to receiving messages
  from a single peer MPI process. Per-peer queues have two mandatory
  sub-parameters ({\em size} and {\em num\_\-buffers}) and three
  optional sub-parameters ({\em low\_\-watermark}, {\em
    window\_\-size}, and {\em reserve}).

\item ``Shared'' receive queues are shared between all MPI sending
  processes. Shared receive queues have the same mandatory
  sub-parameters as per-peer receive queues, but have only two
  optional sub-parameters ({\em low\_\-watermark} and {\em
    max\_\-pending\_\-sends}).
\end{enumerate}

The {\tt btl\_\-openib\_\-receive\_\-queues} value is a
colon-delimited listed of queue specifications specifying the queue
type (``P'' or ``S'') and a comma-delimited list of the mandatory and
optional sub-parameters.  For example:

\vspace{5pt}
\centerline{\tt P,128,256,192,128:S,65535,256,128,32}
\vspace{5pt}

will instantiate one per-peer receive queue for each inbound MPI
connection for messages that are $\le128$ bytes, and will setup a
single shared receive queue for all messages that are $>128$ bytes and
$\le65,535$ bytes (messages longer than 65,535 bytes will be handled
by the long message protocol).

Another good example for how to explore the parameter space by OTPO are the tunable values
controlling Open MPI's use of RDMA for short messages.  Short message
RDMA is a resource-intensive, non-scalable optimization for minimizing
point-to-point short message latency.  Finding a good balance between
the desired level of optimization and the resources consumed by this
optimization is exactly the kind of task that OTPO was designed for. Among the most relevant parameters with regard to RDMA operations are {\tt btl\_\-openib\_\-ib\_\-max\_\-rdma\_\-dst\_\-opts}, which limits the maximum number of outstanding RDMA operations to a specific destination; {\tt btl\_\-openib\_\-use\_\-eager\_\-rdma}, a logical value specifying whether to use the RDMA protocol for eager messages; and {\tt btl\_\-openib\_\-eager\_\-rdma\_\-threshold}, only use RDMA for short messages to a given peer after this number of messages has been received from that peer. Due to space limitations, we will not detail all RDMA parameters or present RDMA results of the according OTPO runs.

%%% Local Variables: 
%%% mode: latex
%%% TeX-master: "paper"
%%% End: 


\subsection{Results}

Tests were run on the shark cluster at the University of Houston.
Shark consists of 24 dual-core 2.2GHz AMD Opteron nodes connected by
4x InfiniBand and Gigabit Ethernet network interconnects.  
The InfiniBand switch is connected to a single HCAs on every node, with an {\tt active\_mtu} of 2048 and an {\tt active\_speed} of 2.5 Gbps. OFED 1.1 is installed on the nodes.  A pre-release version of Open MPI v1.3 was used to generate these results, subversion trunk revision 17198. A nightly snapshot of the trunk was used, and configured with debug disabled. All the tests were run with
{\tt mpi\_leave\_pinner} MCA parameter set to one.

OTPO was used to explore the parameter space of {\tt
  btl\_\-openib\_\-receive\_\-queues} to find a set of values that
yield the lowest half round trip short message latency.  Since {\tt receive\_\-queues}
is a multiple-value parameter, each sub-parameter must be described to
OTPO.  The individual sub-parameters become ``virtual'' parameters,
each with a designated range to explore.  OTPO was configured to test
both a per-peer and a shared receive queue with the ranges listed in
Table~\ref{table:eval-queue-search-params}.  Each sub-parameter
spanned its range by doubling its value from the minimum to the
maximum (e.g., 1, 2, 4, 8, 16, ...).

\def\yes{$\sqrt{}$}

\begin{table}[tb]
\centering
\caption{InfiniBand receive queue search parameter ranges.The ``max
  pending sends'' sub-parameter is only relevant for shared receive
  queues.}
\label{table:eval-queue-search-params} 
\begin{tabular}{|l|c|c|c|} 
\multicolumn{1}{c}{Sub-parameter} &
\multicolumn{1}{c}{Range} &
\multicolumn{1}{c}{Per-peer} &
\multicolumn{1}{c}{Shared} \\
\hline
Buffer size (bytes) & 65,536 $\rightarrow$ 1,048,576 & \yes & \yes \\
Number of buffers & 1 $\rightarrow$ 1024 & \yes & \yes \\
Low watermark (buffers) & 32 $\rightarrow$ 512  & \yes & \yes \\
Max pending sends & 1 $\rightarrow$ 32 & & \yes \\
\hline
\end{tabular}
\end{table}

The parameters that are used are explained as follows:\\
\begin{itemize}
\item The size of the receive buffers to be posted.
\item The maximum number of buffers posted for incoming message fragments.
\item The number of available buffers left on the queue before Open
  MPI reposts buffers up to the maximum (previous parameter).
\item The maximum number of outstanding sends that are allowed at a
  given time (SRQ only).
\end{itemize}

The parameter space from Table~\ref{table:eval-queue-search-params} yields,
275 for per peer queue and 825 for shared queue valid combinations (after
removing unnecessary combinations that would lead to incorrect
results). These combinations stressed buffer management and flow control
issues in the Open MPI short message protocol when sending 1 byte
messages. It took OTPO 3 minutes to evaluate the first
case by invoking NetPIPE for each parameter combination and 9 minutes for the second case.  
Note that NetPIPE runs several ping-pong tests and reports half the average round-trip time. OTPO sought parameter sets that minimized this value.

\begin{table}[tb]
\centering
\caption{OTPO results of the best parameter combinations.}
\label{table:results} 
\begin{tabular}{|c|c||c|c|} 
\hline
\multicolumn{2}{|c|}{PPQ} & \multicolumn{2}{|c|}{SRQ} \\
\hline
Latency & Number of Combinations & Latency & Number of Combinations\\
\hline
3.78$\mu s$  & 3& 3.77$\mu s$  & 1\\ 
\hline		  
3.79$\mu s$  & 3 & 3.78$\mu s$  & 4\\ 
\hline		  
3.80$\mu s$  & 15 & 3.79$\mu s$  & 18\\
\hline
3.81$\mu s$  & 21 & 3.80$\mu s$  & 32\\
\hline		  
3.82$\mu s$  & 31 &3.81$\mu s$  & 69\\
\hline		   
3.83$\mu s$  & 34 & 3.82$\mu s$  & 69\\
\hline
\end{tabular}  
\end{table}


The results are summarized in Table~\ref{table:results}, and reveal a small number of parameter sets that resulted in the lowest latency ($3.78\mu s$ and $3.77\mu s$). However, there were more parameter combinations leading to results within $0.05\mu s$ of the best latency.  These results highlight, that typically, the optimization process using OTPO will not deliver a single set of parameters leading to the best performance, but will result in groups of parameter sets leading to similar performance. %With timings this low, jitter within the results is to be expected.

%%% Local Variables: 
%%% mode: latex
%%% TeX-master: "paper"
%%% End: 
